\documentclass[usenames,dvipsnames]{beamer}

\mode<presentation> {

% The Beamer class comes with a number of default slide themes
% which change the colors and layouts of slides. Below this is a list
% of all the themes, uncomment each in turn to see what they look like.

%\usetheme{default}
% \usetheme{AnnArbor}
%\usetheme{Antibes}
%\usetheme{Bergen}
% \usetheme{Berkeley}
% \usetheme{Berlin}
%\usetheme{Boadilla}
%\usetheme{CambridgeUS}
% \usetheme{Copenhagen}
\usetheme{Darmstadt}
% \usetheme{Dresden}
% \usetheme{Frankfurt}
% \usetheme{Goettingen}
%\usetheme{Hannover}
%\usetheme{Ilmenau}
%\usetheme{JuanLesPins}
%\usetheme{Luebeck}
% \usetheme{Madrid}
% \usetheme{Malmoe}
%\usetheme{Marburg}
% \usetheme{Montpellier}
% \usetheme{PaloAlto}
% \usetheme{Pittsburgh}
%\usetheme{Rochester}
% \usetheme{Singapore}
%\usetheme{Szeged}
%\usetheme{Warsaw}

% As well as themes, the Beamer class has a number of color themes
% for any slide theme. Uncomment each of these in turn to see how it
% changes the colors of your current slide theme.

% \usecolortheme{albatross}
%\usecolortheme{beaver}
%\usecolortheme{beetle}
%\usecolortheme{crane}
%\usecolortheme{dolphin}
%\usecolortheme{dove}
%\usecolortheme{fly}
\usecolortheme{lily}
%\usecolortheme{orchid}
%\usecolortheme{rose}
%\usecolortheme{seagull}
%\usecolortheme{seahorse}
%\usecolortheme{whale}
%\usecolortheme{wolverine}

%\setbeamertemplate{footline} % To remove the footer line in all slides uncomment this line
%\setbeamertemplate{footline}[page number] % To replace the footer line in all slides with a simple slide count uncomment this line

\setbeamertemplate{navigation symbols}{} % To remove the navigation symbols from the bottom of all slides uncomment this line
}

% % -------------------------------------------
% % Songkai added, feel free to delete --------
% \AtBeginSection[]{
%   \begin{frame}
%   \vfill
%   \centering
%   \begin{beamercolorbox}[sep=8pt,center,shadow=false,rounded=true]{title}
%     \usebeamerfont{title}\insertsectionhead\par%
%   \end{beamercolorbox}
%   \vfill
%   \end{frame}
% }
% % Songkai added, feel free to delete --------
% % -------------------------------------------




\usepackage{graphicx} % Allows including images
\usepackage{booktabs} % Allows the use of \toprule, \midrule and \bottomrule in tables
\usepackage{natbib}
\usepackage{amsmath, amssymb, graphicx, url}
\usepackage[ruled]{algorithm2e}
\usepackage{commath}
\usefonttheme[onlymath]{serif}

\usepackage{amsmath}
\usepackage{amssymb}
\usepackage{centernot}
\usepackage{comment}
%\usepackage[a4paper, margin=0.8in]{geometry}
\usepackage{parskip}
\usepackage{graphicx}

\usepackage{natbib}

\usepackage{tikz}
\usepackage{tikzlings}

\usepackage{tabularx}
\usepackage{array}
\usepackage{multirow}
\usepackage{makecell}
\usepackage{mathtools}
\usepackage{bm,upgreek}
\usepackage{subcaption}
\usepackage{textpos}
% \usepackage{eso-pic}

% \usepackage{multimedia}
\usepackage{media9}

\def\E{\mathbf{E}}
\def\PP{\mathbf{P}}
\def\Reals{\mathbb{R}}
\def\Naturals{\mathbb{N}}
\def\argmin{\operatornamewithlimits{arg\,min}}
\def\deq{:=}
\def\wh#1{\widehat{#1}}
\def\bd#1{\mathbf{#1}}
\def\bx{\bd{x}}
\def\by{\bd{y}}
\def\bZ{\bd{Z}}
\def\bB{\bd{B}}
\def\bV{\bd{V}}
\def\tO{{\tilde{\cO}}}
\def\tOm{\tilde{\Omega}}
\def\barw{\overline{w}}
\def\d{{\mathrm d}}
\def\ave#1{\langle #1 \rangle}
\def\Ave#1{\left\langle #1 \right\rangle}
\def\eps{\varepsilon}
\def\tr{\mathrm{Tr}}


\def\HS{\mathbb{H}}
\def\reals{\mathbb{R}}
\def\ths{\theta^*}
\def\thh{\hat{\theta}}
\def\lbr{\left[}
\def\rbr{\right]}
\def\lc{\left(}
\def\rc{\right)}


    \def\ddefloop#1{\ifx\ddefloop#1\else\ddef{#1}\expandafter\ddefloop\fi}
    % \cA, \cB, ...
    \def\ddef#1{\expandafter\def\csname c#1\endcsname{\ensuremath{\mathcal{#1}}}}
    \ddefloop ABCDEFGHIJKLMNOPQRSTUVWXYZ\ddefloop
    \def\argmin{\operatornamewithlimits{arg\,min}}
    \def\E{\mathbf{E}}
    \def\bx{\bd{x}}
	\def\by{\bd{y}}
    \def\bZ{\bd{Z}}

\newcommand{\propnumber}{} % initialize
\newtheorem*{prop}{Proposition \propnumber}
\newenvironment{propc}[1]
  {\renewcommand{\propnumber}{#1}%
   \begin{shaded}\begin{prop}}
  {\end{prop}\end{shaded}}

\newcommand{\crlrnumber}{} % initialize
%\newtheorem*{corollary}{Corollary \crlrnumber}
\newenvironment{corollaryc}[1]
  {\renewcommand{\crlrnumber}{#1}%
   \begin{shaded}\begin{corollary}}
  {\end{corollary}\end{shaded}}

\theoremstyle{definition}
% \newtheorem{definition}




% \setbeamertemplate{headline}{% 
%     \leavevmode%
%     \hbox{%
%         \begin{beamercolorbox}[wd=.4\paperwidth,ht=2.25ex,dp=1ex,right]{section in head/foot}%
%             \usebeamerfont{section in head/foot}\insertshorttitle\hspace*{2ex}
%         \end{beamercolorbox}%
%         \begin{beamercolorbox}[wd=.6\paperwidth,ht=2.25ex,dp=1ex,left]{subsection in head/foot}%
%             \usebeamerfont{section in head/foot}\includegraphics[height=2ex,keepaspectratio]{Slides/Block_M-Hex.png}\hspace*{2ex}\insertsectionhead
%         \end{beamercolorbox}%
%     }
% }

% \addtobeamertemplate{headline}{}{%
% \begin{textblock*}{100mm}(.85\textwidth,-1cm)
% \Huge\textcolor{white}{\textbf{\TeX}}
% \end{textblock*}}



%----------------------------------------------------------------------------------------
%	TITLE PAGE
%----------------------------------------------------------------------------------------

%%%% TRY WITH TEXTPOS
% \newcommand{\imgblock}{\begin{textblock*}{5cm}(10.5cm,-1.2cm) % {block width} (coords)
%         \includegraphics[width=1cm]{Slides/Block_M-Hex.png} % loading the image
%     \end{textblock*}
%     }

% \addtobeamertemplate{background}{\imgblock}{}



% \setbeamertemplate{headline}{\hfill\includegraphics[width=1.5cm]{Slides/Block_M-Hex.png}\hspace{0.2cm}\vspace{-1cm}}

% \logo{\includegraphics[height=1cm]{Slides/Block_M-Hex.png}}

\addtobeamertemplate{frametitle}{}{%
    \begin{textblock*}{5cm}(10.5cm, -0.8cm)
        \includegraphics[width=0.9cm]{Block_M-Hex.png} % your logo file here
    \end{textblock*}
}

\definecolor{mycolor}{cmyk}{100, 60, 0, 60}
\definecolor{my_maize}{rgb}{0.9608,0.7137,0.2588}
\definecolor{my_yellow}{rgb}{0.9294,0.8196,0.2706}

% \setbeamercolor{section in head/foot}{fg=cyan}
\setbeamercolor{section in head/foot}{fg=my_maize}

\setbeamercolor{frametitle}{bg=mycolor}
\setbeamercolor{titlelike}{fg=black, bg=yellow}

\usepackage{url}
\usepackage{hyperref}

\usepackage{xcolor}

\hypersetup{pdfauthor={Name},
            colorlinks=true,
            linkcolor={my_yellow},
            % citecolor={blue},
            % linkcolor=[RGB]{0.949, 0.784, 0.035}
            }

% fix inconsistent colors in cite parenthesis (where the closing parenthesis were black instead of the rest of the citecolor!)
% https://github.com/josephwright/beamer/issues/671
\let\oldcite=\cite
\let\oldcitet=\citet
\let\oldcitep=\citep 
\renewcommand{\citet}[2][]{\textcolor{green}{\oldcitet[#1]{#2}}}
\renewcommand{\citep}[2][]{\textcolor{green}{\oldcitep[#1]{#2}}}
\renewcommand{\cite}[2][]{\textcolor{green}{\oldcite[#1]{#2}}}
            

\title[Seminar]{Towards a Multifidelity Estimate for SVGD}
% \title[Seminar]{An Adaptive Bayesian Method for Covariance Estimation in Multifidelity Estimators}

% and Estimation of Predictive Uncertainties
% The short title appears at the bottom of every slide, the full title is only on the title page

% \author[AJ]{Aniket}
% \institute[U-M]{University of Michigan}

\date{\today}

\AtBeginSection[]
{
 \begin{frame}<beamer>
 \frametitle{Plan}
 \tableofcontents[currentsection]
 \end{frame}
}


\begin{document}

\begin{frame}
\titlepage % Print the title page as the first slide
\end{frame}

% \begin{frame}{Clarification: different initialization, 1-2 particles}


% \end{frame}

% \begin{frame}{Clarification: different initialization, 1-2 particles}
     

% \bigskip 

\begin{frame}
    \textbf{Variational Inference}

    Approximate a target distribution $p(x)$, using a simpler $q^{\ast}(x)$ from a known family, minimize the KL divergence:

    $$q^{\ast} = \arg \min_{q \in \mathcal{Q}}[KL(q || p)]$$

    Desirable properties?!

    - accuracy

    - tractability

    - solvability
\end{frame}

% \bigskip
% % Now, onto other interesting aspects of Stein-flavored math!
% \end{frame}



\begin{frame}{Non-parametric inference}

\textbf{Stein Variational Gradient Descent}
(Liu and Wang 2016)

Transports a set of particles ${x_i}_{i=1}^{n}$ to approximate target $p$ by the update at iteration $l$:

FOR $n$ particles:
    $$x_i^{l+1} \leftarrow x_i^{l} + \epsilon \phi(x_i)$$

What is the optimal choice of the velocity field $\phi$ to push the particles closer to the target distribution?

\end{frame}

\begin{frame}{Deriving the Velocity Field}
    If the particles at current step follow distribution $q_{[\epsilon \phi]}$, then we wish to solve:

    $$\phi^{\ast} = \arg \max_{\phi \in \mathcal{F}}\left\{-\frac{d}{d\epsilon}KL(q_{[\epsilon \phi]} || p)|_{\epsilon=0}\right\}$$

    After going through the weeds of defining the function set $\mathcal{F}$ (RKHS formed by scalar-valued functions associated with PD $k$), the authors use the connection between this objective and Stein's identity:

    \begin{aligned}
        & -\frac{d}{d\epsilon}\mathrm{KL}(q_{[\epsilon\boldsymbol{\phi}]}\parallel p)|_{\epsilon=0}=\mathbb{E}_{x\sim q}[\mathcal{T}_{p}\boldsymbol{\phi}(x)], \\
        & \mathrm{with}\quad\mathcal{T}_{p}\boldsymbol{\phi}(z)=\nabla_{x}\log p(z)^{\top}\boldsymbol{\phi}(z)+\nabla_{z}^{\top}\boldsymbol{\phi}(z)
    \end{aligned}
\end{frame}

\begin{frame}{Stein's Identity}
    Stein's identity supplies that for $\phi$ and a continuously differentiable $p(x)$:

    $$\mathbb{E}_{x\sim p}[\mathcal{T}_{p}\boldsymbol{\phi}(x)] = 0$$

    (any operator satisfying the above is Stein's operator, and $\phi$ is in the Stein class of $p$)

    We end up with a closed form update for $\phi$:

    $$\boldsymbol{\hat{\phi}}^*(x_i^{\ell})=\frac{1}{n}\sum_{i=1}^n (\underbrace{\textcolor{red}{k(x_j^\ell,x_i^\ell)\nabla_{x_j^\ell}\log p(x_j^\ell)}}_{\text{Towards target distribution}}+\underbrace{\textcolor{blue}{\nabla_{x_j^\ell}k(x_j^\ell,x_i^\ell)}}_{\text{Repulsive force keeping particles apart}})$$

\end{frame}

\begin{frame}{Toy Examples}
    \begin{figure}[H]
        \centering
        \includegraphics[width=0.8\linewidth]{GaussianDensity.png}
        \caption{Toy Problem 1- Gaussian Density}        
    \end{figure}
\end{frame}

\begin{frame}{Toy Examples}
    \begin{figure}[H]
        \centering
        \includegraphics[width=0.8\linewidth]{NealsFunnel.png}
        \caption{Toy Problem 2 - Neals Funnel}        
    \end{figure}
\end{frame}


\begin{frame}{No free lunch}
    This works out great, but of course in practice there are scaling issues!

    - All methods that use kernels (kernel-based hypothesis testing, vanilla SVGD) fail eventually as the curse of dimensionality hits! SVGD will end up collapsing to the modes of the target distribution

    - So, most improvements center around - deriving projected transport maps based on active subspaces, message passing variants that solve several lower-dimensional problems and so on.

    Missing pieces:

    - \textcolor{orange}{Dealing with the cost of expensive likelihood updates}

    - Dealing with a suitable choice of kernel, and reducing complexity of the particle-wise force updates.
\end{frame}

\begin{frame}{Approximate Control Variate-based formulation}
Can we use ACVs to exploit correlations between simulators of different fidelities and converge to a target density? \textbf{Maybe}


Current generous assumption:

- Direct access to probability densities associated with each model $p_0(x), p_1(x), \cdots, p_m(x)$ ($p_0$ is our high-fidelity target)

- All models are defined on the same set of parameters, alternately, we can project them to a shared parameter space.

For every particle, the ACV-SVGD update takes the form:

\begin{align}
    \Tilde{\phi}(x_i, z,\alpha,\mathcal{A}) &= 
    \hat{\phi}_{0}(x_i, z_{0})+\sum^{M}_{m=1}\alpha_{m}\left( \hat{\phi}_{m}(x_i, z^{\ast}_{m})-\hat{\mu}_{m}(x_i, z_{m}) \right) \nonumber\\ 
\end{align}

\end{frame}


\begin{frame}{Loosely related works}
    Things that might bear some resemblance to this, but will require more careful reading:

    - Learning CDFs in an ACV framework -  \url{https://arxiv.org/abs/2303.06422}

    - HMC swindles - ``chains targeting slightly different stationary distributions still couple approximately when driven by the same randomness'' - antithetic sampling and control variates - \url{https://proceedings.mlr.press/v108/piponi20a/piponi20a.pdf}

    - Stein control variates - \url{https://arxiv.org/pdf/1710.11198}

    - Generalized Stein's identity for vector-valued control variates - \url{https://proceedings.mlr.press/v202/sun23a/sun23a.pdf}

    
\end{frame}




% \begin{frame}{References}
%   \begin{enumerate}
%     \item Kubokawa, T. Stein's identities and related topics. \url{https://doi.org/10.1007/s42081-023-00239-6}
    
%     \item Kattumannil, S.K. On Stein's Identity and its Application. \url{https://www.isid.ac.in/~statmath/2008/isid200805.pdf}
%   \end{enumerate}
% \end{frame}

% \begin{frame}
    
% \end{frame}


\begin{frame}{Notes}

\end{frame}




    


\begin{frame}{Fun facts}
    (from `A Conversation with Charles Stein' by Morris H. DeGroot)
    \emph{``There are grave difficulties in trying to apply the Bayesian notions to interesting problems because of the difficulty of choosing a prior distribution.''}
    
    % \emph{``Not publishing a lot of interesting results: That results from a certain laziness and perfectionism in the bad sense.''}
      
    Full interview: \url{https://www.jstor.org/stable/2245793?seq=6}
    
    Stein lent his name to a lot of things: Stein's paradox, the James-Stein estimator, Stein's identity, Stein operator, etc.
    
    \url{https://sites.google.com/site/steinsmethod/articles}
    
    % We will talk a bit more about Stein's identity and choice of Stein operator.
\end{frame}


\begin{frame}[allowframebreaks]
    \frametitle{References}
    \bibliographystyle{chicago}

    %\bibliographystyle{IEEEtran}
    \bibliography{references}
\end{frame}

% \section{Backup}
% \begin{frame}{}
% \begin{center}
%     \Large{Thank you!}
% \end{center}
        
% \end{frame}

% Nice 15 minute read on the subject and the philosophy: \url{https://www.stochasticlifestyle.com/how-to-train-interpretable-neural-networks-that-accurately-extrapolate-from-small-data/}

% \end{frame}
% \begin{frame}{Backup 5: Implementation}
% Some other day.
% \end{frame}
\end{document}
