% Beamer Presentation
% LaTeX Template
% Version 1.0 (10/11/12)
%
% This template has been downloaded from:
% http://www.LaTeXTemplates.com
%
% License:
% CC BY-NC-SA 3.0 (http://creativecommons.org/licenses/by-nc-sa/3.0/)
%

%----------------------------------------------------------------------------------------
%	PACKAGES AND THEMES
%----------------------------------------------------------------------------------------

\documentclass[usenames,dvipsnames]{beamer}

\mode<presentation> {

% The Beamer class comes with a number of default slide themes
% which change the colors and layouts of slides. Below this is a list
% of all the themes, uncomment each in turn to see what they look like.

%\usetheme{default}
% \usetheme{AnnArbor}
%\usetheme{Antibes}
%\usetheme{Bergen}
% \usetheme{Berkeley}
% \usetheme{Berlin}
%\usetheme{Boadilla}
%\usetheme{CambridgeUS}
% \usetheme{Copenhagen}
\usetheme{Darmstadt}
% \usetheme{Dresden}
% \usetheme{Frankfurt}
% \usetheme{Goettingen}
%\usetheme{Hannover}
%\usetheme{Ilmenau}
%\usetheme{JuanLesPins}
%\usetheme{Luebeck}
% \usetheme{Madrid}
% \usetheme{Malmoe}
%\usetheme{Marburg}
% \usetheme{Montpellier}
% \usetheme{PaloAlto}
% \usetheme{Pittsburgh}
%\usetheme{Rochester}
% \usetheme{Singapore}
%\usetheme{Szeged}
%\usetheme{Warsaw}

% As well as themes, the Beamer class has a number of color themes
% for any slide theme. Uncomment each of these in turn to see how it
% changes the colors of your current slide theme.

% \usecolortheme{albatross}
%\usecolortheme{beaver}
%\usecolortheme{beetle}
%\usecolortheme{crane}
%\usecolortheme{dolphin}
%\usecolortheme{dove}
%\usecolortheme{fly}
\usecolortheme{lily}
%\usecolortheme{orchid}
%\usecolortheme{rose}
%\usecolortheme{seagull}
%\usecolortheme{seahorse}
%\usecolortheme{whale}
%\usecolortheme{wolverine}

%\setbeamertemplate{footline} % To remove the footer line in all slides uncomment this line
%\setbeamertemplate{footline}[page number] % To replace the footer line in all slides with a simple slide count uncomment this line

\setbeamertemplate{navigation symbols}{} % To remove the navigation symbols from the bottom of all slides uncomment this line
}

% % -------------------------------------------
% % Songkai added, feel free to delete --------
% \AtBeginSection[]{
%   \begin{frame}
%   \vfill
%   \centering
%   \begin{beamercolorbox}[sep=8pt,center,shadow=false,rounded=true]{title}
%     \usebeamerfont{title}\insertsectionhead\par%
%   \end{beamercolorbox}
%   \vfill
%   \end{frame}
% }
% % Songkai added, feel free to delete --------
% % -------------------------------------------




\usepackage{graphicx} % Allows including images
\usepackage{booktabs} % Allows the use of \toprule, \midrule and \bottomrule in tables
\usepackage{natbib}
\usepackage{amsmath, amssymb, graphicx, url}
\usepackage[ruled]{algorithm2e}
\usepackage{commath}
\usefonttheme[onlymath]{serif}

\usepackage{amsmath}
\usepackage{amssymb}
\usepackage{centernot}
%\usepackage[a4paper, margin=0.8in]{geometry}
\usepackage{parskip}
\usepackage{graphicx}

\usepackage{natbib}

\usepackage{tikz}
\usepackage{tikzlings}

\usepackage{tabularx}
\usepackage{array}
\usepackage{multirow}
\usepackage{makecell}
\usepackage{mathtools}
\usepackage{bm,upgreek}
\usepackage{subcaption}
\usepackage{textpos}
% \usepackage{eso-pic}

% \usepackage{multimedia}
\usepackage{media9}

\def\E{\mathbf{E}}
\def\PP{\mathbf{P}}
\def\Reals{\mathbb{R}}
\def\Naturals{\mathbb{N}}
\def\argmin{\operatornamewithlimits{arg\,min}}
\def\deq{:=}
\def\wh#1{\widehat{#1}}
\def\bd#1{\mathbf{#1}}
\def\bx{\bd{x}}
\def\by{\bd{y}}
\def\bZ{\bd{Z}}
\def\bB{\bd{B}}
\def\bV{\bd{V}}
\def\tO{{\tilde{\cO}}}
\def\tOm{\tilde{\Omega}}
\def\barw{\overline{w}}
\def\d{{\mathrm d}}
\def\ave#1{\langle #1 \rangle}
\def\Ave#1{\left\langle #1 \right\rangle}
\def\eps{\varepsilon}
\def\tr{\mathrm{Tr}}


\def\HS{\mathbb{H}}
\def\reals{\mathbb{R}}
\def\ths{\theta^*}
\def\thh{\hat{\theta}}
\def\lbr{\left[}
\def\rbr{\right]}
\def\lc{\left(}
\def\rc{\right)}


    \def\ddefloop#1{\ifx\ddefloop#1\else\ddef{#1}\expandafter\ddefloop\fi}
    % \cA, \cB, ...
    \def\ddef#1{\expandafter\def\csname c#1\endcsname{\ensuremath{\mathcal{#1}}}}
    \ddefloop ABCDEFGHIJKLMNOPQRSTUVWXYZ\ddefloop
    \def\argmin{\operatornamewithlimits{arg\,min}}
    \def\E{\mathbf{E}}
    \def\bx{\bd{x}}
	\def\by{\bd{y}}
    \def\bZ{\bd{Z}}

\newcommand{\propnumber}{} % initialize
\newtheorem*{prop}{Proposition \propnumber}
\newenvironment{propc}[1]
  {\renewcommand{\propnumber}{#1}%
   \begin{shaded}\begin{prop}}
  {\end{prop}\end{shaded}}

\newcommand{\crlrnumber}{} % initialize
%\newtheorem*{corollary}{Corollary \crlrnumber}
\newenvironment{corollaryc}[1]
  {\renewcommand{\crlrnumber}{#1}%
   \begin{shaded}\begin{corollary}}
  {\end{corollary}\end{shaded}}

\theoremstyle{definition}
% \newtheorem{definition}




% \setbeamertemplate{headline}{% 
%     \leavevmode%
%     \hbox{%
%         \begin{beamercolorbox}[wd=.4\paperwidth,ht=2.25ex,dp=1ex,right]{section in head/foot}%
%             \usebeamerfont{section in head/foot}\insertshorttitle\hspace*{2ex}
%         \end{beamercolorbox}%
%         \begin{beamercolorbox}[wd=.6\paperwidth,ht=2.25ex,dp=1ex,left]{subsection in head/foot}%
%             \usebeamerfont{section in head/foot}\includegraphics[height=2ex,keepaspectratio]{Slides/Block_M-Hex.png}\hspace*{2ex}\insertsectionhead
%         \end{beamercolorbox}%
%     }
% }

% \addtobeamertemplate{headline}{}{%
% \begin{textblock*}{100mm}(.85\textwidth,-1cm)
% \Huge\textcolor{white}{\textbf{\TeX}}
% \end{textblock*}}



%----------------------------------------------------------------------------------------
%	TITLE PAGE
%----------------------------------------------------------------------------------------

%%%% TRY WITH TEXTPOS
% \newcommand{\imgblock}{\begin{textblock*}{5cm}(10.5cm,-1.2cm) % {block width} (coords)
%         \includegraphics[width=1cm]{Slides/Block_M-Hex.png} % loading the image
%     \end{textblock*}
%     }

% \addtobeamertemplate{background}{\imgblock}{}



% \setbeamertemplate{headline}{\hfill\includegraphics[width=1.5cm]{Slides/Block_M-Hex.png}\hspace{0.2cm}\vspace{-1cm}}

% \logo{\includegraphics[height=1cm]{Slides/Block_M-Hex.png}}

\addtobeamertemplate{frametitle}{}{%
    \begin{textblock*}{5cm}(10.5cm, -0.8cm)
        \includegraphics[width=0.9cm]{Block_M-Hex.png} % your logo file here
    \end{textblock*}
}

\definecolor{mycolor}{cmyk}{100, 60, 0, 60}
\definecolor{my_maize}{rgb}{0.9608,0.7137,0.2588}
\definecolor{my_yellow}{rgb}{0.9294,0.8196,0.2706}

% \setbeamercolor{section in head/foot}{fg=cyan}
\setbeamercolor{section in head/foot}{fg=my_maize}

\setbeamercolor{frametitle}{bg=mycolor}
\setbeamercolor{titlelike}{fg=black, bg=yellow}

\usepackage{url}
\usepackage{hyperref}

\usepackage{xcolor}

\hypersetup{pdfauthor={Name},
            colorlinks=true,
            linkcolor={my_yellow},
            % citecolor={blue},
            % linkcolor=[RGB]{0.949, 0.784, 0.035}
            }

% fix inconsistent colors in cite parenthesis (where the closing parenthesis were black instead of the rest of the citecolor!)
% https://github.com/josephwright/beamer/issues/671
\let\oldcite=\cite
\let\oldcitet=\citet
\let\oldcitep=\citep 
\renewcommand{\citet}[2][]{\textcolor{green}{\oldcitet[#1]{#2}}}
\renewcommand{\citep}[2][]{\textcolor{green}{\oldcitep[#1]{#2}}}
\renewcommand{\cite}[2][]{\textcolor{green}{\oldcite[#1]{#2}}}
            

\title[Group Meeting]{Parametrized Neural ODEs applied to Field QoIs}
% and Estimation of Predictive Uncertainties
% The short title appears at the bottom of every slide, the full title is only on the title page

\author[Aniket Jivani]{Aniket Jivani}
% \institute[U-M]{University of Michigan}

\date{\today}

\AtBeginSection[]
{
 \begin{frame}<beamer>
 \frametitle{Plan}
 \tableofcontents[currentsection]
 \end{frame}
}


\begin{document}

\begin{frame}
\titlepage % Print the title page as the first slide
\end{frame}



\begin{frame}
 \frametitle{Overview} % Table of contents slide, comment this block out to remove it
 \tableofcontents % Throughout your presentation, if you choose to use \section{} and \subsection{} commands, these will automatically be printed on this slide as an overview of your presentation
\end{frame}

\section{Background}

\begin{frame}{Modeling solutions from PDEs}
    \begin{enumerate}
        \item Cost of forward UQ is very high for simulations from complex systems.
        
        \item We are looking for surrogate models that can learn the dynamics (sometimes from very little data) and help extrapolate evolution of the QoIs better. Notable examples are PINNs, Neural Operators, and UDEs/ NODEs.
        

        \begin{figure}
            \centering
            \includegraphics[width=0.95\linewidth]{SpectrumOfPDESolvingApproaches.png}
            % \caption{Enter Caption}
            \label{fig:spectrum_pde}
        \end{figure}
        
    \end{enumerate}
\end{frame}



\begin{frame}{Advantages of NODE based methods}
    
    \begin{enumerate}
    \item Memory savings - with various tricks can avoid storing intermediate activations to get gradients of loss function. \cite{chen2019neural}
    % as knowing the state at any $t$ we can reconstruct the trajectory forwards and backwards - see algebraic and analytic reversibility in Kidger et al. 2022! (section 5.3.2.1)
    (basic selling point of implicit layer based methods!)
    
    \item Tractable normalizing flows without restrictive architectures by replacing the $|\cdot|$ with a trace approximation (see FFJORD \cite{grathwohl_ffjord:_2018}) (haven't reproduced any detailed experiments for speedups, log-likelihood vs common NF architectures)
    
    \item Fairly simple implementation of a NODE layer in existing DL frameworks with plugin of standardized ODE solvers 
    
    \item More general framework of UDEs \cite{rackauckas_universal_2021} permits us to mix in prior knowledge of some terms and extend to more systematic discovery with sparse regression
    \end{enumerate}
\end{frame}

\begin{frame}{Challenges}
    \begin{enumerate}
        \item Adding latent parameter information to the network (2-3 parameters works fine, but as we increase these it gets harder and harder to differentiate between samples - we saw this for 1D data)
        
        \item Reducing dimensionality of the field data - can embed into encoder style architectures, but we need to explore if training multiple networks this way is tractable, alongside suitable transformations to make efficient embeddings.
        
        \item (Forecasting scenario) Transferring information to a new event. How do we fine-tune the model with a small number of target examples?
    \end{enumerate}
\end{frame}


\begin{frame}{Batch for training PNODEs}
    Two step sampling - sample $n$ simulations at different $\mu$ followed by $m$ ICs from different $t_i$, integrate each IC from each time series up to $k$ timesteps - $n \times m \times k \times (d + p)$ tensor.

\begin{table}[]
\centering
\begin{tabular}{cllll}
                          & $i_1$                    & $i_2$                    & $\cdots$ & $i_m$                    \\
\multirow{3}{*}{$j_1$}        & $y_0^{(i_1)}(\mu_{j_1})$     & $y_0^{(i_2)}(\mu_1)$     &          & $y_0^{(i_m)}(\mu_{j_1})$     \\
                          & $\cdots$                 & $\cdots$                 &          & $\cdots$                 \\
                          & $y_{k-1}^{(i_1)}(\mu_{j_1})$ & $y_{k-1}^{(i_2)}(\mu_{j_1})$ &          & $y_{k-1}^{(i_m)}(\mu_{j_1})$ \\
\multirow{3}{*}{$\vdots$} &                          &                          &          &                          \\
                          &                          & $\vdots$                 &          &                          \\
                          &                          &                          &          &                          \\
\multirow{3}{*}{$j_n$}        & $y_0^{(i_1)}(\mu_{j_n})$     & $y_0^{(i_2)}(\mu_{j_n})$     &          & $y_0^{(i_m)}(\mu_{j_n})$     \\
                          & $\cdots$                 & $\cdots$                 &          & $\cdots$                 \\
                          & $y_{k-1}^{(i_1)}(\mu_{j_n})$ & $y_{k-1}^{(i_2)}(\mu_{j_n})$ &          & $y_{k-1}^{(i_m)}(\mu_{j_n})$
\end{tabular}
% \caption{}
\label{tab: train_tensor}
\end{table}
\end{frame}



    


\section{Architecture}




\section{Toy examples}



\section{White Light Images}
% \begin{frame}{Backup: Bonus}


\begin{frame}[allowframebreaks]
    \frametitle{References}
    \bibliographystyle{chicago}

    %\bibliographystyle{IEEEtran}
    \bibliography{myReference}
\end{frame}

\section{Backup - FML, IL}
\begin{frame}{}
\begin{center}
    \Large{Thank you!}
\end{center}
        
\end{frame}

% Nice 15 minute read on the subject and the philosophy: \url{https://www.stochasticlifestyle.com/how-to-train-interpretable-neural-networks-that-accurately-extrapolate-from-small-data/}

% \end{frame}
% \begin{frame}{Backup 5: Implementation}
% Some other day.
% \end{frame}
\end{document}